%! TeX program = lualatex
%----------------------------------------------------------------------------------------
%    PACKAGES AND THEMES
%----------------------------------------------------------------------------------------

\documentclass[aspectratio=169,xcolor=dvipsnames,10pt]{beamer}
\usetheme{SimplePlus}

\usepackage{hyperref}
\usepackage{graphicx} % Allows including images
\usepackage{booktabs} % Allows the use of \toprule, \midrule and \bottomrule in tables
\usepackage{bbm}
\usepackage{bm}
\usepackage{mathtools}
\usepackage{amsthm}
\usepackage{tikz-cd}
\usepackage{listings}

%%%%%%% JULIA %%%%%%%%%%
\usepackage{fontspec}

\newfontfamily\JuliaMono{JuliaMono}[
	UprightFont = *-Regular,
	BoldFont = *-Bold,
	Path =/Users/davibarreira/Library/Fonts/,
	Extension = .ttf]
\newfontface\JuliaMonoRegular{JuliaMono-Regular}
\newfontface\JuliaMonoBold{JuliaMono-Bold}
% \setmonofont{JuliaMono-Light}[Contextuals=Alternate]


\input{julia_listings}
\input{julia_listings_unicode}

\lstdefinelanguage{JuliaLocal}{
    language = Julia, % inherit Julia lang. to add keywords
}
% \newcommand{\pc}[1]{\lstinline[style=julia]{#1}}
\newcommand{\pc}[1]{\lstinline[style=juliasmall]{#1}}
\newcommand{\pcsmall}[1]{\lstinline[style=juliasmall]{#1}}
\renewcommand{\lstlistingname}{Code}
%%%%%%%%%%%%%%%%%%%%%%%%


\theoremstyle{definition}
\newtheorem{proposition}{Proposition}

\usepackage[square,numbers]{natbib}
\newcommand*{\QEDA}{\hfill\ensuremath{\blacksquare}}%
\newcommand*{\QEDB}{\hfill\ensuremath{\square}}%

%----------------------------------------------------------------------------------------
%    TITLE PAGE
%----------------------------------------------------------------------------------------

\title{Data Visualization From a Category Theory Perspective}
\subtitle{}

\author{Davi Sales Barreira, Asla Medeiros e Sá}

\institute
{
    FGV - EMAp, IMPATech
}
\date{\today} % Date, can be changed to a custom date

%----------------------------------------------------------------------------------------
%    PRESENTATION SLIDES
%----------------------------------------------------------------------------------------
\setbeamertemplate{blocks}[default]
\setbeamercolor{block title}{fg=white, bg=RoyalBlue}
\setbeamercolor{block body}{fg=black, bg=white}

\begin{document}

\begin{frame}
    % Print the title page as the first slide
    \titlepage
\end{frame}

%------------------------------------------------
\begin{frame}{Table of contents}
	% \setbeamertemplate{section in toc}[sections numbered]
	% \setbeamertemplate{subsection in toc}[subsections numbered]
	% \setbeamerfont{subsection in toc}{size=\small}
	% \tableofcontents[sectionstyle=show, subsectionstyle=show]
    \begin{enumerate}
        \item Category Theory in Programming
        \item Julia's Type System
        \item Functional Programming
        \item Categorical Programming in Julia
    \end{enumerate}
\end{frame}

\begin{frame}[fragile]{Category Theory in Programming}
    Perhaps the most influential application of Category Theory has been in
    programming, specially within Functional Programming.
    \citet{orchard2012categorical} states that the application of CT to
    programming can be divided into two distinct approaches, namely
    \textit{categorical programming} and \textit{categorical semantics}.
    \begin{itemize}
        \item Categorical Semantics formally interprets programming languages through the structures of Category Theory;
        \item Categorical Programming uses categorical concepts as design patterns for organizing and structuring programs.
    \end{itemize}
\end{frame}

\begin{frame}[fragile]{Category Theory in Programming}
    In \textbf{Categorical Programming}, programming can be loosely interpreted as a subcategory of $\mathbf{Set}$.
    \begin{itemize}
        \item Sets = Types;
        \item Functions = Programming functions;
        \item Functors = Parametric type with an \pc{fmap} function;
        \item Natural transformations = Parametric polymorphisms.
    \end{itemize}
    ...
\end{frame}

\begin{frame}[fragile]{Julia's Type System}
    Notebook
\end{frame}

\begin{frame}[fragile]{Functional Programming}
    Notebook
\end{frame}

\begin{frame}[fragile]{Categorical Programming in Julia}
    Notebook
\end{frame}

%------------------------------------------------
\begin{frame}{References}
    \footnotesize
    \bibliography{reference.bib}
    \bibliographystyle{apalike}
\end{frame}

%------------------------------------------------

% \begin{frame}
%     \Huge{\centerline{\textbf{The End}}}
% \end{frame}

%----------------------------------------------------------------------------------------

\end{document}
